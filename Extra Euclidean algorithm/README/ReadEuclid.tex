

\documentclass[a4paper,11pt]{article}
\usepackage{latexsym}
\usepackage{amssymb}
%\usepackage{psfrag}
\usepackage[dvips]{graphicx}


\setlength{\topmargin}{-2.0cm}
%\setlength{\textheight}{25.5cm}
\setlength{\textheight}{26cm}
%\setlength{\textwidth}{16cm}
\setlength{\textwidth}{17.5cm}
%%\setlength{\oddsidemargin}{0cm}
\setlength{\oddsidemargin}{-0.5cm}
\setlength{\evensidemargin}{0cm}
\setlength{\parindent}{0pt}
\setlength{\parskip}{4pt} % adjusts the spacing between paragraphs

\begin{document}
%\begin{center}
%Response to referees' comments for GR/T24036/01:
%\end{center}
%
\begin{center}
Euclid's algorithm and the \textsc{Matlab} programme \texttt{Euclid.m}
\end{center}
%
%\begin{center}
%Ana Marco, Jos\'e-Javier Martin\'ez
%\end{center}
\date{}
%

\hrulefill

%\thispagestyle{empty}  % no page number is printed  on this page

\vspace{0.5cm}

\begin{itemize}
\item{
The programme \texttt{Euclid.m} implements Euclid's algorithm for the
Bernstein basis polynomials $f=f(y)$ and $g=g(y)$.}
\end{itemize}

The algorithm is described in the paper:

\textit{Algorithm 812:BPOLY: An object-oriented library of numerical
algorithms for polynomials in Bernstein form}, Y. Tsai and R. Farouki, ACM Transactions on Mathematical Software,
volume 27, number 2, June 2001, pp. 267-296.
%



\vspace{0.5cm}

To run the programme \texttt{Euclid.m}, type
%
\begin{center}
\texttt{Euclid(n,ec,tolerance)}
\end{center}
%
where
%
\begin{description}
\item{\hspace{0.7cm} \texttt{n} is an integer that defines the polynomials $f(y)$ and $g(y)$ in the programme
\texttt{ex.m}}
%
\item{\hspace{0.7cm} \texttt{ec} is the ratio
%
\begin{eqnarray*}
\frac{\textrm{noise level}}{\textrm{signal level}}
\end{eqnarray*}
%
measured in the componentwise sense.}
%
\item{\hspace{0.7cm} \texttt{tolerance} is the stopping criterion for the termination of
Euclid's algorithm}
\end{description}

\vspace{0.25cm}

Examples: Four examples of executing the programme \texttt{Euclid.m} are
%

\texttt{Euclid(8,0,1e-5), Euclid(14,0,1e-5), Euclid(19,1e-8,1e-2), Euclid(20,1e-10,1e-3)}

$\hfill\Box$




Note: The database \texttt{ex.m} is exactly the same as
 the database \texttt{ex.m} for the Sylvester  and B\'ezout matrices.

The method requires that a linear algebraic equation of the form $Ax=b$,
where $A\in \mathbb{R}^{m \times m}$ is non-singular, is solved
at each stage of Euclid's algorithm. This equation is solved by applying the
\textrm{QR} decomposition to the augmented matrix $\left[A\, \,\,b\right]$,
%
\begin{eqnarray*}
\left[A\, \,\,b\right]=QR,\qquad Q\in\mathbb{R}^{m\times m},\quad R^{m \times (m+1)}.
\end{eqnarray*}
%
Since $Ax=b$ can be written as
%
\begin{eqnarray*}
\left[
A \,\, \,b \right]
%
\left[\begin{array}{c} x \\ -1\end{array}\right]=0,
\end{eqnarray*}
%
it follows that
%
\begin{eqnarray*}
QR
%
\left[\begin{array}{c} x \\ -1\end{array}\right]=Q\left[
R_1 \,\, \,r  \right]
%
\left[\begin{array}{c} x \\ -1\end{array}\right]=0,
\end{eqnarray*}
%
where $R_1\in\mathbb{R}^{m\times m}$ is upper triangular and $r$ is the
$(m+1)$th column of $R$.
Since $Q$ is non-singular, the equation $Ax=b$ is transformed to the
equation $R_1x=r$.



\end{document}



      

