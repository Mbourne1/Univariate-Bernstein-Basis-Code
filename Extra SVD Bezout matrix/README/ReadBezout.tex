

\documentclass[a4paper,11pt]{article}
\usepackage{latexsym}
\usepackage{amssymb}
%\usepackage{psfrag}
\usepackage[dvips]{graphicx}


\setlength{\topmargin}{-2.0cm}
%\setlength{\textheight}{25.5cm}
\setlength{\textheight}{26cm}
%\setlength{\textwidth}{16cm}
\setlength{\textwidth}{17.5cm}
%%\setlength{\oddsidemargin}{0cm}
\setlength{\oddsidemargin}{-0.5cm}
\setlength{\evensidemargin}{0cm}
\setlength{\parindent}{0pt}
\setlength{\parskip}{4pt} % adjusts the spacing between paragraphs

\begin{document}
%\begin{center}
%Response to referees' comments for GR/T24036/01:
%\end{center}
%
\begin{center}
The B\'ezout resultant matrix and the \textsc{Matlab} programme \texttt{o.m}
\end{center}
%
%\begin{center}
%Ana Marco, Jos\'e-Javier Martin\'ez
%\end{center}
\date{}
%

\hrulefill

%\thispagestyle{empty}  % no page number is printed  on this page

\vspace{0.5cm}

\begin{itemize}
\item{
The programme \texttt{o.m}  applies the singular value decomposition
to the Bezout matrix $B(f,g)$
 of the
Bernstein basis polynomials $f=f(y)$ and $g=g(y)$ with $\theta=1$, and
the optimal value of $\theta$ computed by solving a linear programming problem.}
%
\end{itemize}


\vspace{0.5cm}

To run the programme \texttt{o.m}, type
%
\begin{center}
\texttt{o(n,ec)}
\end{center}
%
where
%
\begin{description}
\item{\hspace{0.7cm} \texttt{n} is an integer that defines the polynomials $f(y)$ and $g(y)$ in the programme
\texttt{ex.m}}
%
\item{\hspace{0.7cm} \texttt{ec} is the ratio
%
\begin{eqnarray*}
\frac{\textrm{noise level}}{\textrm{signal level}}
\end{eqnarray*}
%
measured in the componentwise sense.}
\end{description}

\vspace{0.25cm}

Examples: Three examples of executing the programme \texttt{o.m} are
%

\texttt{o(15,0), o(23,1e-8), o(30,1e-8)}

$\hfill\Box$


\vspace{0.25cm}

The programme produces four  graphs:
%
\begin{enumerate}
\item{Figure 1 shows the column sums of $B(f,g)$ for $\theta=1$.}
\item{Figure 2 shows the column sums of $B(f,g)$ for $\theta=\theta_0$,
where $\theta_0$ is the optimal value of $\theta$.}
\item{Figure 3 shows the normalised singular values of $B(f,g)$ for $\theta=1$.}
\item{Figure 4 shows the normalised singular values of $B(f,g)$ for $\theta=\theta_0$.}
\end{enumerate}

Note: The database \texttt{ex.m} is exactly the same as
 the database \texttt{ex.m} for the other programmes on
 Bernstein polynomials.



\end{document}



        

