

\documentclass[a4paper,11pt]{article}
\usepackage{latexsym}
\usepackage{amssymb}
\usepackage{amsmath}
\usepackage[acronym]{glossaries}
%\usepackage{psfrag}
\usepackage[dvips]{graphicx}

\newacronym{AGCD}{AGCD}{Approximate Greatest Common Divisor}
\newacronym{APF}{APF}{Approximate Polynomial Factorisation}
\newacronym{CAGD}{CAGD}{Computer Aided Geometric Design}
\newacronym{GCD}{GCD}{Greatest Common Divisor}
\newacronym{SVD}{SVD}{Singular Value Decomposition}
\newacronym{SNTLN}{SNTLN}{Structured Non-Linear Total Least Norm}
\newacronym{CAD}{CAD}{Computer Aided Design}

%New commands
\newcommand{\GCD}{\textrm{GCD}}
\newcommand{\AGCD}{\textrm{AGCD}}
\newcommand{\ML}{\textrm{ML}}
\newcommand{\MDL}{\textrm{MDL}}
\newcommand{\rank}{\textrm{rank}}
\newcommand{\STLN}{\textrm{STLN}}
\newcommand{\SNTLN}{\textrm{SNTLN}}
\newcommand{\LSE}{\textrm{LSE}}
\newcommand{\SVD}{\textrm{SVD}}
\newcommand{\QR}{\textrm{QR}}
\newcommand{\APF}{\textrm{APF}}


\setlength{\topmargin}{-2.0cm}
%\setlength{\textheight}{25.5cm}
\setlength{\textheight}{26cm}
%\setlength{\textwidth}{16cm}
\setlength{\textwidth}{17.5cm}
%%\setlength{\oddsidemargin}{0cm}
\setlength{\oddsidemargin}{-0.5cm}
\setlength{\evensidemargin}{0cm}
\setlength{\parindent}{0pt}
\setlength{\parskip}{4pt} % adjusts the spacing between paragraphs

\begin{document}
%\begin{center}
%Response to referees' comments for GR/T24036/01:
%\end{center}
%
\begin{center}
Degree of an $\AGCD$ of two Bernstein polynomials
\end{center}
%
\date{}
%

\hrulefill

%\thispagestyle{empty}  % no page number is printed  on this page

\vspace{0.5cm}

This note describes the programs
%
\begin{center}
\texttt{o\_roots\_Univariate.m} 
and 
\texttt{o\_gcd\_Univariate\_2Polys.m}
\end{center}
%
\begin{enumerate}

	\item The first file is used in the computation of a polynomial's roots and corresponding multiplicities, where the given polynomial is in Bernstein form.
	
	\item The second file is used in teh computation of the \gls{GCD} of two polynomials in Bernstein form.


\end{enumerate}


The programs are executed by typing
%
%
\begin{center}
\texttt{o\_roots\_Univariate(ex\_num, emin, emax, mean\_method, bool\_alpha\_theta, low\_rank\_approx\_method, apf\_method, sylvester\_build\_method)}

\texttt{o\_gcd\_2Polys\_Univariate(ex\_num, emin, emax, mean\_method, bool\_alpha\_theta, low\_rank\_approx\_method, apf\_method, sylvester\_build\_method)}
\end{center}
%
 where
%
\begin{description}

	\item[\texttt{ex\_num}]
	A String typically containing an integer, which defines the example to be run.
	
	\item[\texttt{emin}] Minimum signal : noise ratio
	
	\item[\texttt{emax}] Maximum signal : noise ratio
	
	\item[\texttt{mean\_method}]
	Method used to compute the mean of the entries of the partitions of the Sylvester subresultant matrix
		\begin{description}
			\item[\texttt{None}] No mean method used
			\item[\texttt{Geometric Mean My Method}] : Fast method 
			\item[\texttt{Geometric Mean Matlab Method}] : Standard Matlab method
			\item[\texttt{Arithmetic Mean}] : 
		\end{description} 

	
	\item 
	\texttt{bool\_alpha\_theta} 
		\begin{description}
			\item[\texttt{true}] : Preprocess polynomials
			\item[\texttt{false}] : Exclude preprocessing 
		\end{description}


	
	\item 
	\texttt{low\_rank\_approx\_method}
		\begin{description}
			\item[\texttt{None}] 
			\item[\texttt{Standard SNTLN}] 
			\item[\texttt{Standard STLN}]
			\item[\texttt{Root Specific SNTLN}] 
		\end{description}
	
	
	
	\item 
	\texttt{apf\_method}
		\begin{description}
			\item[\texttt{None}] 
			\item[\texttt{Standard Linear APF}]
			\item[\texttt{Standard NonLinear APF}]
		\end{description}


	\item
	\texttt{Sylvester\_Build\_Method}
		\begin{description}
			
			\item[\texttt{T} : ]
			The matrix 
			$
				T_{k}
				\left(
					f(x)
					,
					g(x)
				\right)
			$
			
			
			\item[\texttt{DT} : ]
			The matrix 
			$
				D^{-1}_{m+n-k}
				T_{k}
				\left(
					f(x)
					,
					g(x)
				\right)
			$
			
			\item[\texttt{DTQ} : ]
			The matrix
			$
				D^{-1}_{m+n-k}
				T_{k}
				\left(
					f
					,
					g
				\right)
				\hat{Q}
			$
			
			
			\item[\texttt{TQ} : ]
			The matrix
			$
				T_{k}
				\left(
					f
					,
					g
				\right)
				\hat{Q}
			$
			
			\item[\texttt{DTQ Rearranged} : ]
			The matrix 
			$
				S_{k}
				\left(
					f(x)
					,
					g(x)
				\right)
				=
				D^{-1}_{m+n-k}
				T_{k}
				\left(
					f(x)
					,
					g(x)
				\right)
				\hat{Q}
			$ where the entries are computed in a rearranged form.
			
			\item[\texttt{DTQ Rearranged Denom Removed} : ]
			$
				\tilde{S}
				\left(
					f(x)
					,
					g(x).
				\right)
			$
			
		\end{description}

%
\end{description}

Examples of executing the programs  are
%

\texttt{o\_gcd\_Univariate\_2Polys('1', 1e-10, 1e-12, 'Geometric Mean Matlab Method', true, 'None', 'None', 'DTQ')}

\texttt{o\_roots\_Univariate('1', 1e-12, 1e-10, 'Geometric Mean Matlab Method', true, 'None', 'None', 'DTQ')}


The programs produce the following output:
%
\section{Points of Interest}
\subsection{Limits}
The code makes frequent use of variables \texttt{t\_limits} and \texttt{k\_limits}. 
\begin{description}

	\item[\texttt{t\_limits} : ] In the computation of the factorisation of $\hat{f}_{0}(x)$, many \gls{GCD} computations are required to generate the sequence 
	$
		\hat{f}_{i}(x)
		=
		GCD
		\left(
			\hat{f}_{i-1}(x)
			,
			\hat{f}_{i-1}^{'}(x)
		\right)
	$. The degree of
	$
		GCD
		\left(
			\hat{f}_{i}(x)
			,
			\hat{f}_{i}^{'}(x)
		\right)
	$ is bound by the number of distinct roots of $\hat{f}_{i}(x)$, and the number of distinct roots is always less than or equal to the number of distinct roots of $\hat{f}_{i-1}(x)$. 
	
	\item[\texttt{k\_limits} : ] This variable defines the range of Sylvester subresultant matrices 
	$
		S_{k}
		\left(
			\hat{f}_{i}(x)
			,
			\hat{f}_{i}^{'}(x)
		\right)
	$ considered in the computation of the degree of the \gls{GCD}. By default this range is set between $1$ and $\min(m,n)$, however \texttt{limits\_t} can also be used since it is known that all subresultant matrices outside this range are known to be singular. 
\end{description}

\subsection{Computing the Degree of the GCD}
There are several methods considered for the computation of the degree of the \gls{GCD}. A global variable \texttt{SETTINGS.RANK\_REVEALING\_METRIC} defined in the file \texttt{SetGlobalVariables.m} determines which method is used.

\begin{description}
	
	\item[Singular Values : ] 
	
	\item[R1 Row Norms :]
	
	\item[R1 Row Diagonals : ]
	
	\item[Residuals : ]
	
\end{description}

%



\end{document}





