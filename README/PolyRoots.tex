

\documentclass[a4paper,11pt]{article}
\usepackage{latexsym}
\usepackage{amssymb}
\usepackage{amsmath}
%\usepackage{psfrag}
\usepackage[dvips]{graphicx}

%New commands
\newcommand{\GCD}{\textrm{GCD}}
\newcommand{\AGCD}{\textrm{AGCD}}
\newcommand{\ML}{\textrm{ML}}
\newcommand{\MDL}{\textrm{MDL}}
\newcommand{\rank}{\textrm{rank}}
\newcommand{\STLN}{\textrm{STLN}}
\newcommand{\SNTLN}{\textrm{SNTLN}}
\newcommand{\LSE}{\textrm{LSE}}
\newcommand{\SVD}{\textrm{SVD}}
\newcommand{\QR}{\textrm{QR}}
\newcommand{\APF}{\textrm{APF}}


\setlength{\topmargin}{-2.0cm}
%\setlength{\textheight}{25.5cm}
\setlength{\textheight}{26cm}
%\setlength{\textwidth}{16cm}
\setlength{\textwidth}{17.5cm}
%%\setlength{\oddsidemargin}{0cm}
\setlength{\oddsidemargin}{-0.5cm}
\setlength{\evensidemargin}{0cm}
\setlength{\parindent}{0pt}
\setlength{\parskip}{4pt} % adjusts the spacing between paragraphs

\begin{document}
%\begin{center}
%Response to referees' comments for GR/T24036/01:
%\end{center}
%
\begin{center}
Degree of an $\AGCD$ of two Bernstein polynomials
\end{center}
%
\date{}
%

\hrulefill

%\thispagestyle{empty}  % no page number is printed  on this page

\vspace{0.5cm}

This note describes the programs
%
\begin{center}
\texttt{o\_roots.m} and \texttt{o\_gcd.m}
\end{center}
%
The first is for calculating the roots of a polynomial in the Bernstein basis. The second is for calculating the GCD of two arbitrary polynomials. Both of these files use \texttt{o1.m} to calculate a $\GCD$ and quotient polynomials. 
 Bernstein polynomials $f(y)$ and $g(y)$. The modified matrix
$S(f,g)$.

The programs are executed by typing
%
%
\begin{center}
\texttt{o\_roots(n,ec\_min,ec\_max,BOOL\_SNTLN,BOOL\_APF,BOOL\_DENOM,BOOL\_PREPROC,BOOL\_LOG)} and \\
\texttt{o\_gcd(n,ec\_min,ec\_max,BOOL\_SNTLN,BOOL\_APF,BOOL\_DENOM,BOOL\_PREPROC,BOOL\_LOG)}
\end{center}
%
 where
%
\begin{description}
\item{\hspace{0.7cm} 
\texttt{n} is an integer that defines the polynomials $f(y)$ in the program
\texttt{Root\_examples.m}} or polynomials $f(y)$ and $g(y)$ in the program \texttt{GCD\_examples.m} for \texttt{o\_roots} and \texttt{o\_gcd.m} respectively.
%
\item{\hspace{0.7cm} \texttt{ec\_min} and \texttt{ec\_max} are the minimum and maximum ratios:


%
\begin{align*}
\frac{\textrm{noise level}}{\textrm{signal level}}
\end{align*}
%
measured in the componentwise sense.}
\item{\hspace{0.7cm}
\texttt{BOOL\_SNTLN} is a boolean value, whether structured perturbations are added to the Sylvester matrix.
}
\texttt{BOOL\_APF} is a boolean value, whether structured perturbations are added to the Approximate polynomial factorisation.
}

\end{description}
%
The differences in the programs are:
\begin{itemize}
\item{
The program \text{o\_gcd.m} takes two polynomials $f(y)$ and $g(y)$ and calculates the $\GCD$ of their perturbed forms.
}
\item{
The program \text{o\_roots.m} takes only one exact polynomial $\hat{f}(y)$ adds noise to its coefficients such that $f(y)$ is the noisy form, calculates it's derivative $f'(y)$ and performs a $\gcd$ calculation to obtain $d_{1}(y)$. A new $\gcd$ calculation is performed on d(y) and its derivative. This is performed iteratively until the GCD calculation output is a scalar. The set of gcd values $d_{1},d_{2},\dots d_n$ are used to compute the roots and corresponding multiplicities of $\hat{f}(y)$
}
%

\end{itemize}

\vspace{0.5cm}    


Examples of executing the programs  are
%

\texttt{o1(5,1e-8), o2(15,1e-7), o3(21,1e-5)} and \texttt{o4(30,1e-7)}.

$\hfill\Box$



The programs produce the following output:
%

%

%



\end{document}





